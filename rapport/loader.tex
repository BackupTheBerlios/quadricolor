\section{Loader}

Le loader, comme son nom l'indique, est la partie du projet qui se
charge de r�cup�rer un objet sur le disque physique et de l'envoyer au
cache.
Le loader est appel� par le cache lorsque le visualisateur demande �
ce dernier une ressource dont il ne dispose pas en m�moire, mais aussi
si le cache a besoin de conna�tre la taille physique qu'un fichier
image occupe sur le disque dur.\\

Afin de charger et visualiser les images, nous avons h�sit� pendant un
moment entre les classes de la librairie Magick++ et celles de Qt.
Notre choix s'est finalement port� sur Qt car en plus de la
compatibilit� avec le visualisateur (lui aussi r�alis� avec Qt), les
librairies QImage et QFile de Qt nous permettaient d'obtenir assez
rapidement les r�sultats escompt�s qui sont:

\begin{itemize}
\item Charger une image depuis le disque physique en vue la visualiser
\item Obtenir la taille physique du fichier image
\end{itemize}

Ce r�sultat peut �tre aussi atteint en utilisant la classe Image de
Magick++, mais cette classe �tant plus adapt� pour le traitement de
l'image (zoom, redimensionnement de l'image...), la compilation �tait
plus longue qu'avec Qt.

Nous avons cependant implant� un Loader g�n�rique pouvant fonctionner
avec l'une ou l'autre librairie en fonction de la classe qui est
pass�e dans le template suivant : 

{\policecommande template<class K, class Image_Type, class File_Type>}

\subsection{Chargement d'un objet}

Pour charger l'objet, nous avons besoin d'une cl� K permettant de
l'identifier. Dans le cadre de ce projet, la cl� est le chemin d'acc�s
(path + nom) du fichier image. Si la cl� fournie est invalide ou ne
correspond pas � un fichier image, la m�thode de chargement l�ve une
exception: {\policecommande ImageNotFoundException}.
Sinon on retourne un objet de type Image_Type (QImage, dans le cadre
de ce projet).

M�thode � appel�: getObject. \\

\subsection{Taille physique d'un objet}

Comme pour le chargement de l'objet, nous avons besoin d'une cl� K
pour l'identifier. Nous pouvons donc cr�er un objet de type File_Type
nous permettant d'obtenir la taille du fichier sous la forme d'un
entier que l'on retourne.

M�thode � appel�: getSize. \\






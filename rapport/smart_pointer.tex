\section{Le smart pointer}
Salut je me pr�sente, je suis en Maitrise d'informatique et je suis �
la recherche d'un travail renumer� avec vacances garanties, en
pr�vision de mon DESS que je n'aurais pas sachant qu'on aura sera en
concurrence avec des brels qui auront de meilleures moyennes parce
qu'ils auront le temps de terminer le peu de projets.
Et � quoi bon faire un stage si on est � plein temps � la fac.

Un Smart Pointer est un objet qui encapsule un pointeur et qui teste
les acc�s en m�moire sur cet objet. Mais malgr� tout sa manipulation
reste parfois ambig�e (differente de celles d'un vrai pointeur) \\par ex:
{\policecommande \\void *p;\\SmartPointer sp1(p);\\SmartPointer
*sp2(p);\\return (*sp1==*(*sp2);//renvoit true }

Un Smart Poiter est compos� de 2 parties : le Smart Pointeur en tant
que tel (la partie qui gere les acc�s) et un Reference Counter qui
peut �tre partag� par plusieurs Smart Pointer.

\subsection{Presentation des classes}
Toutes les classes � l'exception d'une se trouve dans le package Pointer.
Nous avons 2 classes de Smart Pointer:
\begin{itemize}
\item StandardSmartPointer (prot�ge les acc�s en levant une exception,
offre des acc�s synchronis�s � l'aide de mutex, et offre les
fonctionnalit�es de base d'un Smart Pointer),
\item et TabularSmartPointer.
\end{itemize}
2 classes de Reference Counter:
\begin{itemize}
\item DefaultReferenceCounter,
\item et CacheReferenceCounter (dans le package CacheSystem).
\end{itemize}
et 2 exceptions:
\begin{itemize}
\item NullPointerException,
\item et ArrayIndexOutOfBoundsException.
\end{itemize}



